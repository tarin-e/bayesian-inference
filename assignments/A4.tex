% Options for packages loaded elsewhere
\PassOptionsToPackage{unicode}{hyperref}
\PassOptionsToPackage{hyphens}{url}
%
\documentclass[
]{article}
\usepackage{amsmath,amssymb}
\usepackage{iftex}
\ifPDFTeX
  \usepackage[T1]{fontenc}
  \usepackage[utf8]{inputenc}
  \usepackage{textcomp} % provide euro and other symbols
\else % if luatex or xetex
  \usepackage{unicode-math} % this also loads fontspec
  \defaultfontfeatures{Scale=MatchLowercase}
  \defaultfontfeatures[\rmfamily]{Ligatures=TeX,Scale=1}
\fi
\usepackage{lmodern}
\ifPDFTeX\else
  % xetex/luatex font selection
\fi
% Use upquote if available, for straight quotes in verbatim environments
\IfFileExists{upquote.sty}{\usepackage{upquote}}{}
\IfFileExists{microtype.sty}{% use microtype if available
  \usepackage[]{microtype}
  \UseMicrotypeSet[protrusion]{basicmath} % disable protrusion for tt fonts
}{}
\makeatletter
\@ifundefined{KOMAClassName}{% if non-KOMA class
  \IfFileExists{parskip.sty}{%
    \usepackage{parskip}
  }{% else
    \setlength{\parindent}{0pt}
    \setlength{\parskip}{6pt plus 2pt minus 1pt}}
}{% if KOMA class
  \KOMAoptions{parskip=half}}
\makeatother
\usepackage{xcolor}
\usepackage[margin=1in]{geometry}
\usepackage{color}
\usepackage{fancyvrb}
\newcommand{\VerbBar}{|}
\newcommand{\VERB}{\Verb[commandchars=\\\{\}]}
\DefineVerbatimEnvironment{Highlighting}{Verbatim}{commandchars=\\\{\}}
% Add ',fontsize=\small' for more characters per line
\usepackage{framed}
\definecolor{shadecolor}{RGB}{248,248,248}
\newenvironment{Shaded}{\begin{snugshade}}{\end{snugshade}}
\newcommand{\AlertTok}[1]{\textcolor[rgb]{0.94,0.16,0.16}{#1}}
\newcommand{\AnnotationTok}[1]{\textcolor[rgb]{0.56,0.35,0.01}{\textbf{\textit{#1}}}}
\newcommand{\AttributeTok}[1]{\textcolor[rgb]{0.13,0.29,0.53}{#1}}
\newcommand{\BaseNTok}[1]{\textcolor[rgb]{0.00,0.00,0.81}{#1}}
\newcommand{\BuiltInTok}[1]{#1}
\newcommand{\CharTok}[1]{\textcolor[rgb]{0.31,0.60,0.02}{#1}}
\newcommand{\CommentTok}[1]{\textcolor[rgb]{0.56,0.35,0.01}{\textit{#1}}}
\newcommand{\CommentVarTok}[1]{\textcolor[rgb]{0.56,0.35,0.01}{\textbf{\textit{#1}}}}
\newcommand{\ConstantTok}[1]{\textcolor[rgb]{0.56,0.35,0.01}{#1}}
\newcommand{\ControlFlowTok}[1]{\textcolor[rgb]{0.13,0.29,0.53}{\textbf{#1}}}
\newcommand{\DataTypeTok}[1]{\textcolor[rgb]{0.13,0.29,0.53}{#1}}
\newcommand{\DecValTok}[1]{\textcolor[rgb]{0.00,0.00,0.81}{#1}}
\newcommand{\DocumentationTok}[1]{\textcolor[rgb]{0.56,0.35,0.01}{\textbf{\textit{#1}}}}
\newcommand{\ErrorTok}[1]{\textcolor[rgb]{0.64,0.00,0.00}{\textbf{#1}}}
\newcommand{\ExtensionTok}[1]{#1}
\newcommand{\FloatTok}[1]{\textcolor[rgb]{0.00,0.00,0.81}{#1}}
\newcommand{\FunctionTok}[1]{\textcolor[rgb]{0.13,0.29,0.53}{\textbf{#1}}}
\newcommand{\ImportTok}[1]{#1}
\newcommand{\InformationTok}[1]{\textcolor[rgb]{0.56,0.35,0.01}{\textbf{\textit{#1}}}}
\newcommand{\KeywordTok}[1]{\textcolor[rgb]{0.13,0.29,0.53}{\textbf{#1}}}
\newcommand{\NormalTok}[1]{#1}
\newcommand{\OperatorTok}[1]{\textcolor[rgb]{0.81,0.36,0.00}{\textbf{#1}}}
\newcommand{\OtherTok}[1]{\textcolor[rgb]{0.56,0.35,0.01}{#1}}
\newcommand{\PreprocessorTok}[1]{\textcolor[rgb]{0.56,0.35,0.01}{\textit{#1}}}
\newcommand{\RegionMarkerTok}[1]{#1}
\newcommand{\SpecialCharTok}[1]{\textcolor[rgb]{0.81,0.36,0.00}{\textbf{#1}}}
\newcommand{\SpecialStringTok}[1]{\textcolor[rgb]{0.31,0.60,0.02}{#1}}
\newcommand{\StringTok}[1]{\textcolor[rgb]{0.31,0.60,0.02}{#1}}
\newcommand{\VariableTok}[1]{\textcolor[rgb]{0.00,0.00,0.00}{#1}}
\newcommand{\VerbatimStringTok}[1]{\textcolor[rgb]{0.31,0.60,0.02}{#1}}
\newcommand{\WarningTok}[1]{\textcolor[rgb]{0.56,0.35,0.01}{\textbf{\textit{#1}}}}
\usepackage{graphicx}
\makeatletter
\def\maxwidth{\ifdim\Gin@nat@width>\linewidth\linewidth\else\Gin@nat@width\fi}
\def\maxheight{\ifdim\Gin@nat@height>\textheight\textheight\else\Gin@nat@height\fi}
\makeatother
% Scale images if necessary, so that they will not overflow the page
% margins by default, and it is still possible to overwrite the defaults
% using explicit options in \includegraphics[width, height, ...]{}
\setkeys{Gin}{width=\maxwidth,height=\maxheight,keepaspectratio}
% Set default figure placement to htbp
\makeatletter
\def\fps@figure{htbp}
\makeatother
\setlength{\emergencystretch}{3em} % prevent overfull lines
\providecommand{\tightlist}{%
  \setlength{\itemsep}{0pt}\setlength{\parskip}{0pt}}
\setcounter{secnumdepth}{-\maxdimen} % remove section numbering
\ifLuaTeX
  \usepackage{selnolig}  % disable illegal ligatures
\fi
\IfFileExists{bookmark.sty}{\usepackage{bookmark}}{\usepackage{hyperref}}
\IfFileExists{xurl.sty}{\usepackage{xurl}}{} % add URL line breaks if available
\urlstyle{same}
\hypersetup{
  hidelinks,
  pdfcreator={LaTeX via pandoc}}

\author{}
\date{\vspace{-2.5em}}

\begin{document}

\hypertarget{assignment-4}{%
\subsubsection{Assignment 4}\label{assignment-4}}

\hypertarget{tarin-eccleston}{%
\paragraph{Tarin Eccleston}\label{tarin-eccleston}}

1a) Both models E and U imply prior predictive distributions for the
data, and hence the data mean x\_bar. Would the two prior predictive
distributions for̄ x\_bar be the same or different? Explain your answer.

Both E and U will have different prior predictive distributions (ppd)
for x\_bar since both sampling distributions for E and U are different
with exponential and uniform distributions respectively. The prior
distributions remain the same. The general ppd formula shown below.

\[
p(\bar{x}) =  \int_{0}^{\infty} (\prod_{i = 1}^{5} p(x_i| \theta)) p(\theta) d\theta
\]

1b) Part (a) implies that learning only x\_bar would provide some
information about whether E or U is true. Does this seem reasonable to
you?

Yes it does. If the main difference between E or U is the choice of
uniform vs exponential sampling distributions, and we are given x\_bar,
then we can evaluate the posterior distribution in terms of x\_bar. E or
U with the highest posterior distribution would hence be the most true
model.

1c) Write down analytical expressions for the marginal likelihoods p(x
\textbar{} U ) and p(x \textbar{} E). Retain all constant factors.

\[
p(\bar{x} | U) =  \frac{1}{\sqrt{2\pi}} \int_{x_{max}}^{\infty} (\prod_{i = 1}^{5}\frac{1}{b})\frac{1}{b} e^{-\frac{1}{2} (\ln{b})^2} db
\]

Likelihood function is not 1/b for all x.

\[
p(\bar{x} | E) =  \frac{1}{\sqrt{2\pi}} \int_{0}^{\infty} \lambda^4 e^{-(\frac{1}{2} (\ln{\lambda})^2 + 5\lambda\bar{x})} d\lambda
\] 1d) Numerically find the values of the two marginal likelihoods.

\hypertarget{output-code-for-model-u}{%
\subsubsection{Output code for Model U}\label{output-code-for-model-u}}

Marginal likelihood: ln(Z) = -0.3483763 +- 0.05943507. Information: H =
0.3532527 nats. Effective posterior sample size = 337. Posterior samples
saved in ns-posterior-samples.csv.

\hypertarget{model-u-ns-model}{%
\subsubsection{Model U ns-model}\label{model-u-ns-model}}

\begin{Shaded}
\begin{Highlighting}[]
\NormalTok{num\_params }\OtherTok{=} \DecValTok{1}

\NormalTok{parameter\_names }\OtherTok{=} \FunctionTok{c}\NormalTok{(}\StringTok{"theta"}\NormalTok{)}

\NormalTok{data }\OtherTok{=} \FunctionTok{list}\NormalTok{(}\AttributeTok{x=}\FunctionTok{c}\NormalTok{(}\FloatTok{0.610164901707321}\NormalTok{, }\FloatTok{1.99984208494425}\NormalTok{, }\FloatTok{1.50817369576544}\NormalTok{, }\FloatTok{0.707493807654828}\NormalTok{,}
                           \FloatTok{1.49413506453857}\NormalTok{), }\AttributeTok{N=}\DecValTok{5}\NormalTok{)}

\NormalTok{us\_to\_params }\OtherTok{=} \ControlFlowTok{function}\NormalTok{(us)}
\NormalTok{\{}
\NormalTok{    params }\OtherTok{=} \FunctionTok{rep}\NormalTok{(}\ConstantTok{NA}\NormalTok{, num\_params)}

    \FunctionTok{names}\NormalTok{(params) }\OtherTok{=}\NormalTok{ parameter\_names}

\NormalTok{    params[}\StringTok{"theta"}\NormalTok{] }\OtherTok{=} \FunctionTok{exp}\NormalTok{(}\FunctionTok{qnorm}\NormalTok{(us[}\DecValTok{1}\NormalTok{], }\DecValTok{0}\NormalTok{, }\DecValTok{1}\NormalTok{))}

    \FunctionTok{return}\NormalTok{(params)}
\NormalTok{\}}

\NormalTok{log\_likelihood }\OtherTok{=} \ControlFlowTok{function}\NormalTok{(params)}
\NormalTok{\{}
\NormalTok{    logL }\OtherTok{=} \FunctionTok{sum}\NormalTok{(}\FunctionTok{dunif}\NormalTok{(}\AttributeTok{x =}\NormalTok{ data}\SpecialCharTok{$}\NormalTok{x, }\AttributeTok{min =} \DecValTok{0}\NormalTok{, }\AttributeTok{max =}\NormalTok{ params[}\StringTok{"theta"}\NormalTok{], }\AttributeTok{log =} \ConstantTok{TRUE}\NormalTok{)[}\FunctionTok{is.finite}\NormalTok{(}\FunctionTok{dunif}\NormalTok{(}\AttributeTok{x =}\NormalTok{ data}\SpecialCharTok{$}\NormalTok{x, }\AttributeTok{min =} \DecValTok{0}\NormalTok{, }\AttributeTok{max =}\NormalTok{ params[}\StringTok{"theta"}\NormalTok{], }\AttributeTok{log =} \ConstantTok{TRUE}\NormalTok{))])}
    \FunctionTok{return}\NormalTok{(logL)}
\NormalTok{\}}
\end{Highlighting}
\end{Shaded}

\hypertarget{output-code-for-model-e}{%
\subsubsection{Output code for Model E}\label{output-code-for-model-e}}

Marginal likelihood: ln(Z) = -7.059105 +- 0.07194591. Information: H =
0.5176214 nats. Effective posterior sample size = 356. Posterior samples
saved in ns-posterior-samples.csv.

\hypertarget{model-e-ns-model}{%
\subsubsection{Model E ns-model}\label{model-e-ns-model}}

\begin{Shaded}
\begin{Highlighting}[]
\NormalTok{num\_params }\OtherTok{=} \DecValTok{1}

\NormalTok{parameter\_names }\OtherTok{=} \FunctionTok{c}\NormalTok{(}\StringTok{"theta"}\NormalTok{)}

\NormalTok{data }\OtherTok{=} \FunctionTok{list}\NormalTok{(}\AttributeTok{x=}\FunctionTok{c}\NormalTok{(}\FloatTok{0.610164901707321}\NormalTok{, }\FloatTok{1.99984208494425}\NormalTok{, }\FloatTok{1.50817369576544}\NormalTok{, }\FloatTok{0.707493807654828}\NormalTok{,}
                           \FloatTok{1.49413506453857}\NormalTok{), }\AttributeTok{N=}\DecValTok{5}\NormalTok{)}

\NormalTok{us\_to\_params }\OtherTok{=} \ControlFlowTok{function}\NormalTok{(us)}
\NormalTok{\{}
\NormalTok{    params }\OtherTok{=} \FunctionTok{rep}\NormalTok{(}\ConstantTok{NA}\NormalTok{, num\_params)}

    \FunctionTok{names}\NormalTok{(params) }\OtherTok{=}\NormalTok{ parameter\_names}

\NormalTok{    params[}\StringTok{"theta"}\NormalTok{] }\OtherTok{=} \FunctionTok{exp}\NormalTok{(}\FunctionTok{qnorm}\NormalTok{(us[}\DecValTok{1}\NormalTok{], }\DecValTok{0}\NormalTok{, }\DecValTok{1}\NormalTok{))}

    \FunctionTok{return}\NormalTok{(params)}
\NormalTok{\}}

\NormalTok{log\_likelihood }\OtherTok{=} \ControlFlowTok{function}\NormalTok{(params)}
\NormalTok{\{}
\NormalTok{    logL }\OtherTok{=} \FunctionTok{sum}\NormalTok{(}\FunctionTok{dexp}\NormalTok{(}\AttributeTok{x =}\NormalTok{ data}\SpecialCharTok{$}\NormalTok{x, }\AttributeTok{rate =}\NormalTok{ params[}\StringTok{"theta"}\NormalTok{], }\AttributeTok{log =} \ConstantTok{TRUE}\NormalTok{))}
    \FunctionTok{return}\NormalTok{(logL)}
\NormalTok{\}}
\end{Highlighting}
\end{Shaded}

\begin{Shaded}
\begin{Highlighting}[]
\NormalTok{Z\_U }\OtherTok{=} \FunctionTok{exp}\NormalTok{(}\SpecialCharTok{{-}}\FloatTok{0.3483763}\NormalTok{)}
\NormalTok{Z\_U}
\end{Highlighting}
\end{Shaded}

\begin{verbatim}
## [1] 0.7058332
\end{verbatim}

\begin{Shaded}
\begin{Highlighting}[]
\NormalTok{Z\_E }\OtherTok{=} \FunctionTok{exp}\NormalTok{(}\SpecialCharTok{{-}}\FloatTok{7.059105}\NormalTok{)}
\NormalTok{Z\_E}
\end{Highlighting}
\end{Shaded}

\begin{verbatim}
## [1] 0.000859547
\end{verbatim}

1e)

Find the Bayes Factor (either way around) and also the posterior
probabilities of U and E, assuming prior probabilities of 1/2 each.

\begin{Shaded}
\begin{Highlighting}[]
\CommentTok{\# Bayes factor : U/E}
\NormalTok{bayes\_factor }\OtherTok{=}\NormalTok{  Z\_U }\SpecialCharTok{/}\NormalTok{ Z\_E}
\NormalTok{bayes\_factor}
\end{Highlighting}
\end{Shaded}

\begin{verbatim}
## [1] 821.1688
\end{verbatim}

The uniform distribution is \textasciitilde800 times more likely to
produce observed data compared to exponential distribution.

Prior probabilities for U and E: 1/2

\begin{Shaded}
\begin{Highlighting}[]
\NormalTok{prior\_probs }\OtherTok{=} \FunctionTok{c}\NormalTok{(}\DecValTok{1}\SpecialCharTok{/}\DecValTok{2}\NormalTok{, }\DecValTok{1}\SpecialCharTok{/}\DecValTok{2}\NormalTok{)}
\NormalTok{lik }\OtherTok{=} \FunctionTok{c}\NormalTok{(}\SpecialCharTok{{-}}\FloatTok{0.3483763}\NormalTok{, }\SpecialCharTok{{-}}\FloatTok{7.059105}\NormalTok{)}

\NormalTok{h }\OtherTok{=}\NormalTok{ prior\_probs }\SpecialCharTok{*}\NormalTok{ lik}
\NormalTok{Z }\OtherTok{=} \FunctionTok{sum}\NormalTok{(h)}

\NormalTok{posterior\_odds }\OtherTok{=}\NormalTok{ h}\SpecialCharTok{/}\NormalTok{Z}
\NormalTok{posterior\_odds}
\end{Highlighting}
\end{Shaded}

\begin{verbatim}
## [1] 0.04703033 0.95296967
\end{verbatim}

1f) If p(b \textbar{} U ) were made much wider, the Bayes Factor would
strongly favour E. Explain why this occurs.

We have limited data therefore our posterior distributions are sensitive
to the choice of prior. If our uniform prior margins for p(b \textbar{}
U ) widen, we effectively decrease the prior probabilities for the same
region which overlap the likelihood. We are also widen the likelihood
since b is the standard deviation.

As a result our posterior probabilities for p(b \textbar{} U, x)
decreases. Hence our Bayes factor then decreases and favours E instead
of U.

2a) Implement Model A for Nested Sampling, and run it. In your answer,
provide the following: (i) num\_params, (ii), parameter\_names, (iii)
us\_to\_params(), (iv) log\_likelihood(), and (v) the estimated value of
the marginal likelihood P (y \textbar{} A) and its uncertainty.

\hypertarget{output-code-for-model-a}{%
\subsubsection{Output code for Model A}\label{output-code-for-model-a}}

Marginal likelihood: ln(Z) = -685.036 +- 0.3414809. Information: H =
11.66092 nats. Effective posterior sample size = 594. Posterior samples
saved in ns-posterior-samples.csv.

\hypertarget{model-a-ns-model}{%
\subsubsection{Model A ns-model}\label{model-a-ns-model}}

\begin{Shaded}
\begin{Highlighting}[]
\NormalTok{num\_params }\OtherTok{=} \DecValTok{3}

\NormalTok{parameter\_names }\OtherTok{=} \FunctionTok{c}\NormalTok{(}\StringTok{"beta\_0"}\NormalTok{, }\StringTok{"beta\_1"}\NormalTok{, }\StringTok{"sigma"}\NormalTok{)}

\NormalTok{colorado\_df }\OtherTok{=} \FunctionTok{read.csv}\NormalTok{(}\StringTok{"../data/colorado.csv"}\NormalTok{)}

\NormalTok{us\_to\_params }\OtherTok{=} \ControlFlowTok{function}\NormalTok{(us)}
\NormalTok{\{}
\NormalTok{    params }\OtherTok{=} \FunctionTok{rep}\NormalTok{(}\ConstantTok{NA}\NormalTok{, num\_params)}

    \FunctionTok{names}\NormalTok{(params) }\OtherTok{=}\NormalTok{ parameter\_names}

\NormalTok{    params[}\StringTok{"beta\_0"}\NormalTok{] }\OtherTok{=} \FunctionTok{exp}\NormalTok{(}\FunctionTok{qnorm}\NormalTok{(us[}\DecValTok{1}\NormalTok{], }\DecValTok{0}\NormalTok{, }\DecValTok{5}\NormalTok{))}
\NormalTok{    params[}\StringTok{"beta\_1"}\NormalTok{] }\OtherTok{=} \FunctionTok{qnorm}\NormalTok{(us[}\DecValTok{2}\NormalTok{], }\DecValTok{0}\NormalTok{, params[}\StringTok{"beta\_0"}\NormalTok{])}
\NormalTok{    params[}\StringTok{"sigma"}\NormalTok{] }\OtherTok{=} \FunctionTok{exp}\NormalTok{(}\FunctionTok{qnorm}\NormalTok{(us[}\DecValTok{3}\NormalTok{], }\FunctionTok{log}\NormalTok{(params[}\StringTok{"beta\_0"}\NormalTok{]),}\DecValTok{1}\NormalTok{))}

    \FunctionTok{return}\NormalTok{(params)}
\NormalTok{\}}

\NormalTok{log\_likelihood }\OtherTok{=} \ControlFlowTok{function}\NormalTok{(params)}
\NormalTok{\{}
\NormalTok{    mu }\OtherTok{=}\NormalTok{ params[}\StringTok{"beta\_0"}\NormalTok{] }\SpecialCharTok{+}\NormalTok{ params[}\StringTok{"beta\_1"}\NormalTok{]}\SpecialCharTok{*}\NormalTok{colorado\_df}\SpecialCharTok{$}\NormalTok{t}
\NormalTok{    logL }\OtherTok{=} \FunctionTok{sum}\NormalTok{(}\FunctionTok{dnorm}\NormalTok{(}\AttributeTok{x =}\NormalTok{ colorado\_df}\SpecialCharTok{$}\NormalTok{y, }\AttributeTok{mean =}\NormalTok{ mu, }\AttributeTok{sd =}\NormalTok{ params[}\StringTok{"sigma"}\NormalTok{], }\AttributeTok{log =} \ConstantTok{TRUE}\NormalTok{))}
    \FunctionTok{return}\NormalTok{(logL)}
\NormalTok{\}}
\end{Highlighting}
\end{Shaded}

\hypertarget{marginal-likelihood}{%
\paragraph{Marginal Likelihood}\label{marginal-likelihood}}

\begin{Shaded}
\begin{Highlighting}[]
\NormalTok{Z\_MA\_mu }\OtherTok{=} \FunctionTok{exp}\NormalTok{(}\SpecialCharTok{{-}}\FloatTok{685.036}\NormalTok{)}
\NormalTok{Z\_MA\_mu}
\end{Highlighting}
\end{Shaded}

\begin{verbatim}
## [1] 3.109176e-298
\end{verbatim}

\hypertarget{marginal-likelihood-error-range}{%
\paragraph{Marginal Likelihood Error
Range}\label{marginal-likelihood-error-range}}

\begin{Shaded}
\begin{Highlighting}[]
\NormalTok{Z\_MA\_error }\OtherTok{=} \FunctionTok{c}\NormalTok{(}\FunctionTok{exp}\NormalTok{(}\SpecialCharTok{{-}}\FloatTok{685.036} \SpecialCharTok{{-}} \FloatTok{0.3414809}\NormalTok{), }\FunctionTok{exp}\NormalTok{(}\SpecialCharTok{{-}}\FloatTok{685.036} \SpecialCharTok{+} \FloatTok{0.3414809}\NormalTok{))}
\NormalTok{Z\_MA\_error}
\end{Highlighting}
\end{Shaded}

\begin{verbatim}
## [1] 2.209744e-298 4.374703e-298
\end{verbatim}

\hypertarget{output-code-for-model-b}{%
\subsubsection{Output code for Model B}\label{output-code-for-model-b}}

Marginal likelihood: ln(Z) = -683.9028 +- 0.366068. Information: H =
13.40058 nats. Effective posterior sample size = 712. Posterior samples
saved in ns-posterior-samples.csv.

\hypertarget{model-b-ns-model}{%
\subsubsection{Model B ns-model}\label{model-b-ns-model}}

\begin{Shaded}
\begin{Highlighting}[]
\NormalTok{num\_params }\OtherTok{=} \DecValTok{4}

\NormalTok{parameter\_names }\OtherTok{=} \FunctionTok{c}\NormalTok{(}\StringTok{"beta\_0"}\NormalTok{, }\StringTok{"beta\_1"}\NormalTok{, }\StringTok{"beta\_2"}\NormalTok{, }\StringTok{"sigma"}\NormalTok{)}

\NormalTok{colorado\_df }\OtherTok{=} \FunctionTok{read.csv}\NormalTok{(}\StringTok{"../data/colorado.csv"}\NormalTok{)}

\NormalTok{us\_to\_params }\OtherTok{=} \ControlFlowTok{function}\NormalTok{(us)}
\NormalTok{\{}
\NormalTok{    params }\OtherTok{=} \FunctionTok{rep}\NormalTok{(}\ConstantTok{NA}\NormalTok{, num\_params)}

    \FunctionTok{names}\NormalTok{(params) }\OtherTok{=}\NormalTok{ parameter\_names}

\NormalTok{    params[}\StringTok{"beta\_0"}\NormalTok{] }\OtherTok{=} \FunctionTok{exp}\NormalTok{(}\FunctionTok{qnorm}\NormalTok{(us[}\DecValTok{1}\NormalTok{], }\DecValTok{0}\NormalTok{, }\DecValTok{5}\NormalTok{))}
\NormalTok{    params[}\StringTok{"beta\_1"}\NormalTok{] }\OtherTok{=} \FunctionTok{qnorm}\NormalTok{(us[}\DecValTok{2}\NormalTok{], }\DecValTok{0}\NormalTok{, params[}\StringTok{"beta\_0"}\NormalTok{])}
\NormalTok{    params[}\StringTok{"beta\_2"}\NormalTok{] }\OtherTok{=} \FunctionTok{qnorm}\NormalTok{(us[}\DecValTok{3}\NormalTok{], }\DecValTok{0}\NormalTok{, params[}\StringTok{"beta\_0"}\NormalTok{])}
\NormalTok{    params[}\StringTok{"sigma"}\NormalTok{] }\OtherTok{=} \FunctionTok{exp}\NormalTok{(}\FunctionTok{qnorm}\NormalTok{(us[}\DecValTok{4}\NormalTok{], }\FunctionTok{log}\NormalTok{(params[}\StringTok{"beta\_0"}\NormalTok{]),}\DecValTok{1}\NormalTok{))}

    \FunctionTok{return}\NormalTok{(params)}
\NormalTok{\}}

\NormalTok{log\_likelihood }\OtherTok{=} \ControlFlowTok{function}\NormalTok{(params)}
\NormalTok{\{}
\NormalTok{    mu }\OtherTok{=}\NormalTok{ params[}\StringTok{"beta\_0"}\NormalTok{] }\SpecialCharTok{+}\NormalTok{ params[}\StringTok{"beta\_1"}\NormalTok{]}\SpecialCharTok{*}\NormalTok{colorado\_df}\SpecialCharTok{$}\NormalTok{t }\SpecialCharTok{+}\NormalTok{ (colorado\_df}\SpecialCharTok{$}\NormalTok{t }\SpecialCharTok{\textgreater{}=} \FloatTok{0.875}\NormalTok{)}\SpecialCharTok{*}\NormalTok{params[}\StringTok{"beta\_2"}\NormalTok{]}\SpecialCharTok{*}\NormalTok{(colorado\_df}\SpecialCharTok{$}\NormalTok{t }\SpecialCharTok{{-}} \FloatTok{0.875}\NormalTok{)}
\NormalTok{    logL }\OtherTok{=} \FunctionTok{sum}\NormalTok{(}\FunctionTok{dnorm}\NormalTok{(}\AttributeTok{x =}\NormalTok{ colorado\_df}\SpecialCharTok{$}\NormalTok{y, }\AttributeTok{mean =}\NormalTok{ mu, }\AttributeTok{sd =}\NormalTok{ params[}\StringTok{"sigma"}\NormalTok{], }\AttributeTok{log =} \ConstantTok{TRUE}\NormalTok{))}
    \FunctionTok{return}\NormalTok{(logL)}
\NormalTok{\}}
\end{Highlighting}
\end{Shaded}

\hypertarget{marginal-likelihood-1}{%
\paragraph{Marginal Likelihood}\label{marginal-likelihood-1}}

\begin{Shaded}
\begin{Highlighting}[]
\NormalTok{Z\_MB\_mu }\OtherTok{=} \FunctionTok{exp}\NormalTok{(}\SpecialCharTok{{-}}\FloatTok{683.9028}\NormalTok{)}
\NormalTok{Z\_MB\_mu}
\end{Highlighting}
\end{Shaded}

\begin{verbatim}
## [1] 9.65579e-298
\end{verbatim}

\hypertarget{marginal-likelihood-error-range-1}{%
\paragraph{Marginal Likelihood Error
Range}\label{marginal-likelihood-error-range-1}}

\begin{Shaded}
\begin{Highlighting}[]
\NormalTok{Z\_MB\_error }\OtherTok{=} \FunctionTok{c}\NormalTok{(}\FunctionTok{exp}\NormalTok{(}\SpecialCharTok{{-}}\FloatTok{683.9028} \SpecialCharTok{{-}} \FloatTok{0.366068}\NormalTok{), }\FunctionTok{exp}\NormalTok{(}\SpecialCharTok{{-}}\FloatTok{683.9028} \SpecialCharTok{+} \FloatTok{0.366068}\NormalTok{))}
\NormalTok{Z\_MB\_error}
\end{Highlighting}
\end{Shaded}

\begin{verbatim}
## [1] 6.695862e-298 1.392416e-297
\end{verbatim}

\hypertarget{output-code-for-model-c}{%
\subsubsection{Output code for Model C}\label{output-code-for-model-c}}

Marginal likelihood: ln(Z) = -680.6652 +- 0.3902254. Information: H =
15.22758 nats. Effective posterior sample size = 700. Posterior samples
saved in ns-posterior-samples.csv.

\hypertarget{model-c-ns-model}{%
\subsubsection{Model C ns-model}\label{model-c-ns-model}}

\begin{Shaded}
\begin{Highlighting}[]
\NormalTok{num\_params }\OtherTok{=} \DecValTok{4}

\NormalTok{parameter\_names }\OtherTok{=} \FunctionTok{c}\NormalTok{(}\StringTok{"beta\_0"}\NormalTok{, }\StringTok{"beta\_1"}\NormalTok{, }\StringTok{"beta\_2"}\NormalTok{, }\StringTok{"sigma"}\NormalTok{)}

\NormalTok{colorado\_df }\OtherTok{=} \FunctionTok{read.csv}\NormalTok{(}\StringTok{"../data/colorado.csv"}\NormalTok{)}

\NormalTok{us\_to\_params }\OtherTok{=} \ControlFlowTok{function}\NormalTok{(us)}
\NormalTok{\{}
\NormalTok{    params }\OtherTok{=} \FunctionTok{rep}\NormalTok{(}\ConstantTok{NA}\NormalTok{, num\_params)}

    \FunctionTok{names}\NormalTok{(params) }\OtherTok{=}\NormalTok{ parameter\_names}

\NormalTok{    params[}\StringTok{"beta\_0"}\NormalTok{] }\OtherTok{=} \FunctionTok{exp}\NormalTok{(}\FunctionTok{qnorm}\NormalTok{(us[}\DecValTok{1}\NormalTok{], }\DecValTok{0}\NormalTok{, }\DecValTok{5}\NormalTok{))}
\NormalTok{    params[}\StringTok{"beta\_1"}\NormalTok{] }\OtherTok{=} \FunctionTok{qnorm}\NormalTok{(us[}\DecValTok{2}\NormalTok{], }\DecValTok{0}\NormalTok{, params[}\StringTok{"beta\_0"}\NormalTok{])}
\NormalTok{    params[}\StringTok{"beta\_2"}\NormalTok{] }\OtherTok{=} \FunctionTok{qnorm}\NormalTok{(us[}\DecValTok{3}\NormalTok{], }\DecValTok{0}\NormalTok{, params[}\StringTok{"beta\_0"}\NormalTok{])}
\NormalTok{    params[}\StringTok{"sigma"}\NormalTok{] }\OtherTok{=} \FunctionTok{exp}\NormalTok{(}\FunctionTok{qnorm}\NormalTok{(us[}\DecValTok{4}\NormalTok{], }\FunctionTok{log}\NormalTok{(params[}\StringTok{"beta\_0"}\NormalTok{]),}\DecValTok{1}\NormalTok{))}

    \FunctionTok{return}\NormalTok{(params)}
\NormalTok{\}}

\NormalTok{log\_likelihood }\OtherTok{=} \ControlFlowTok{function}\NormalTok{(params)}
\NormalTok{\{}
\NormalTok{    mu }\OtherTok{=}\NormalTok{ params[}\StringTok{"beta\_0"}\NormalTok{] }\SpecialCharTok{+}\NormalTok{ params[}\StringTok{"beta\_1"}\NormalTok{]}\SpecialCharTok{*}\NormalTok{colorado\_df}\SpecialCharTok{$}\NormalTok{t }\SpecialCharTok{+}\NormalTok{ params[}\StringTok{"beta\_2"}\NormalTok{]}\SpecialCharTok{*}\NormalTok{(colorado\_df}\SpecialCharTok{$}\NormalTok{t)}\SpecialCharTok{\^{}}\DecValTok{2}
\NormalTok{    logL }\OtherTok{=} \FunctionTok{sum}\NormalTok{(}\FunctionTok{dnorm}\NormalTok{(}\AttributeTok{x =}\NormalTok{ colorado\_df}\SpecialCharTok{$}\NormalTok{y, }\AttributeTok{mean =}\NormalTok{ mu, }\AttributeTok{sd =}\NormalTok{ params[}\StringTok{"sigma"}\NormalTok{], }\AttributeTok{log =} \ConstantTok{TRUE}\NormalTok{))}
    \FunctionTok{return}\NormalTok{(logL)}
\NormalTok{\}}
\end{Highlighting}
\end{Shaded}

\hypertarget{marginal-likelihood-2}{%
\paragraph{Marginal Likelihood}\label{marginal-likelihood-2}}

\begin{Shaded}
\begin{Highlighting}[]
\NormalTok{Z\_MC\_mu }\OtherTok{=} \FunctionTok{exp}\NormalTok{(}\SpecialCharTok{{-}}\FloatTok{680.6652}\NormalTok{)}
\NormalTok{Z\_MC\_mu}
\end{Highlighting}
\end{Shaded}

\begin{verbatim}
## [1] 2.459572e-296
\end{verbatim}

\hypertarget{marginal-likelihood-error-range-2}{%
\paragraph{Marginal Likelihood Error
Range}\label{marginal-likelihood-error-range-2}}

\begin{Shaded}
\begin{Highlighting}[]
\NormalTok{Z\_MC\_error }\OtherTok{=} \FunctionTok{c}\NormalTok{(}\FunctionTok{exp}\NormalTok{(}\SpecialCharTok{{-}}\FloatTok{680.6652} \SpecialCharTok{{-}} \FloatTok{0.3902254}\NormalTok{), }\FunctionTok{exp}\NormalTok{(}\SpecialCharTok{{-}}\FloatTok{680.6652} \SpecialCharTok{+} \FloatTok{0.3902254}\NormalTok{))}
\NormalTok{Z\_MC\_error}
\end{Highlighting}
\end{Shaded}

\begin{verbatim}
## [1] 1.664895e-296 3.633560e-296
\end{verbatim}

2d) If the three models have prior probabilities P (A) = P (B) = P (C) =
1/3, find the super-duper-marginalised likelihood P (y \textbar{} A ∨ B
∨ C) and the three posterior probabilities.

this is the same as p(y)

\begin{Shaded}
\begin{Highlighting}[]
\NormalTok{prior }\OtherTok{=} \FunctionTok{c}\NormalTok{(}\DecValTok{1}\SpecialCharTok{/}\DecValTok{3}\NormalTok{, }\DecValTok{1}\SpecialCharTok{/}\DecValTok{3}\NormalTok{, }\DecValTok{1}\SpecialCharTok{/}\DecValTok{3}\NormalTok{)}
\NormalTok{marg\_lik }\OtherTok{=} \FunctionTok{c}\NormalTok{(}\FunctionTok{exp}\NormalTok{(}\SpecialCharTok{{-}}\FloatTok{685.036}\NormalTok{), }\FunctionTok{exp}\NormalTok{(}\SpecialCharTok{{-}}\FloatTok{683.9028}\NormalTok{), }\FunctionTok{exp}\NormalTok{(}\SpecialCharTok{{-}}\FloatTok{680.6652}\NormalTok{))}

\NormalTok{h\_super }\OtherTok{=}\NormalTok{ prior }\SpecialCharTok{*}\NormalTok{ marg\_lik}
\NormalTok{Z\_super }\OtherTok{=} \FunctionTok{sum}\NormalTok{(h\_super)}
\NormalTok{Z\_super}
\end{Highlighting}
\end{Shaded}

\begin{verbatim}
## [1] 8.624074e-297
\end{verbatim}

\begin{Shaded}
\begin{Highlighting}[]
\NormalTok{posterior\_probabilities }\OtherTok{=}\NormalTok{ h\_super }\SpecialCharTok{/}\NormalTok{ Z\_super}
\NormalTok{posterior\_probabilities}
\end{Highlighting}
\end{Shaded}

\begin{verbatim}
## [1] 0.01201743 0.03732107 0.95066150
\end{verbatim}

\end{document}
